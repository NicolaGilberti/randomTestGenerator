\section{Primitive input generator}\label{sec:pig}
The process of the generator works hand by hand with the \emph{Test Generator}~\ref{sec:tg}, and it does multiple works.
The goal of this chapter is to define all the primitives types this project can random-generate and how values already instantiated could be reused during their lifecycle.\\
This work is based on the Java language, so its well-known primitives are maintained.
There is an extra types  directly generated from the code and it is the \emph{String}.\\
About the Java primitives, they can be used in a reflection environment only thanks to their object representations like \emph{Integer} for \emph{int}, \emph{Character} for \emph{char} and so on.\\
When the program require the generation of a new variable, and the process find that it require a primitive, it start instantiating the proper types with a default constructor, assigning a fixed value.
Then it is checked if an older generated value exist and, in case of a true response, a first random choice is performed.\\
With an equal probability, it is possible that the program try to generate a new value or choose to pick an old ones.
There is a third options for object types that is the \emph{null} value, but has a smaller probability to appear.\\
When an older value is chosen, the data to link the two references are arranged for future steps.
On the other case, a random value is required, and the Random~\cite{rand} class enter in action.\\
For some primitives, this dependency is able to generate directly a value, while in other cases an extra work is necessary.
In fact, in case of int, float, long, double, and boolean, there are already functions.
\begin{lstlisting}[caption={Random default generator},label={lst:genDef}]% Start your code-block

case "java.lang.Integer":
	obj.setValue(random.nextInt());
case "java.lang.Float":
	obj.setValue(random.nextFloat());
case "java.lang.Long":
	obj.setValue(random.nextLong());
case "java.lang.Double":
	obj.setValue(random.nextDouble());
case "java.lang.Boolean":
	obj.setValue((random.nextBoolean());
\end{lstlisting}
However, in case of short, the random-generator can be done thinking on how the type is constructed.
\begin{lstlisting}[caption={Random short generator},label={lst:genShort}]% Start your code-block
	
case "java.lang.Short":
	obj.setValue(random.nextInt(65536) - 32768);
\end{lstlisting}
Char and string require more operations, thus an extra function enter in action.
\begin{lstlisting}[caption={Random String/char generator},label={lst:genString&Char}]% Start your code-block

public static String generateRandomChars(String candidateChars, int length, Random random) {
	StringBuilder sb = new StringBuilder();
	for (int i = 0; i < length; i++) {
		sb.append(candidateChars.charAt(random.nextInt(candidateChars.length())));
	}
	return sb.toString();
}
\end{lstlisting}
The difference between String and char is the length input, which is 1 for character, and random generated for strings.
Variable \emph{candidateChars} is a sequence of accepted character. 
For example, \emph{"ABCDEFGHIJKLMNOPQRSTUVWXYZ1234567890abcdefhijklmnopqrstuvwxyz"} can be used if upper/lower case letters and numbers are the only accepted values we want to generate with.\\
In the Java language, there are also different types of objects, and one of them is the Enumerator~\cite{enum}.\\
This program consider Enums as a primitive type and when one is found, it generate one instance of it with a random value accordingly with the requirement of randomness.\\
With this particular object type, the equal operation work differently and so, the links for the usage of old values are not necessary. 
\begin{lstlisting}[caption={Equality in enums},label={lst:enumEquality}]% Start your code-block

Enum v1 = Enum.value1;
Enum v2 = Enum.value1;
System.out.println( v1 == v2 );
//True
\end{lstlisting}
In case of it is required an instantiation that do not belong to the previous cases, a recursive approach is adopted.
\begin{lstlisting}[caption={non primitive/enums recursion},label={lst:recursion}]% Start your code-block

int maxVal = targetClass.getConstructors().length;
int ran = random.nextInt(maxVal);
Constructor<?> targetC = targetClass.getConstructors()[ran];
int parVal = targetC.getParameterTypes().length;
if(parVal!=0) {
	try {
		req = new Object[parVal];
		Class<?>[] paramType = targetC.getParameterTypes();
		Object[] par = istantiatedArray(req, paramType);
		obj[i] = Class.forName(targetClassName).getConstructor(paramType).newInstance(par);
	}catch(NoSuchMethodException  | IllegalArgumentException | InvocationTargetException e) {
		System.err.println("Instantiator error, it will try again " + e);
	}
}else {
	obj[i] = Class.forName(targetClassName).newInstance();
}
\end{lstlisting}
The 'unknown' class is analysed, collecting all its constructors.
Then one is randomly picked and, if variables are required to execute the constructor, the random generation process is re-called on each element, line 9 in~\ref{lst:recursion}.
If the constructor does not require any input variables, Java reflection can instantiate it directly.\\
There is also the option in which the class was already instantiated with random values and the randomness choose to take an old option instead of a new one.
This is automatically done as in the case of primitives, and the links between variables are saved accordingly.
For example, there is a class Clazz that has to be instantiated.
It is nor primitive neither an enum, so a random constructor is picked.
The one taken require two variables, an int and another unknown class Clazz2.
So the generator is recalled on the int variable and on the Clazz2.
About the primitive typos, the program could generate a new value or pick an old ones if exist.
Clazz2 has only the default empty constructor, thus Reflection can instantiate it directly.
Now there is the backtrack on the recursion and the Clazz can be instantiated with the constructor chosen and the input variables created just before.\\
With this approach the program is able to instantiate any kind of object, which are referenced in the environment, that are required for tests.\\
As an example of the environment requirement, in case of a class X is required to be instantiated and its source is not in the class-path, the instantiation fail and the creation of tests \emph{is} compromised.