\chapter{Empirical evaluation} \label{ch:EmpEvaluation}
This chapter focus on the evaluation of the proposed algorithm, making a comparison between this and the Evosuite method.
The comparison is done executing 10 time both programs on the applications and analyzing the output in terms of coverage and number of test executed to define the final testsuite.
The evaluation is performed by statistical tests.\\
The distribution of the result cannot be assessed, caused by the random approach followed, so a t-test can't be performed.\\
However, the \emph{Wilcoxon rank sum test}, also known as Mann–Whitney U test, avoid this limitation.
It quantifies significant differences between the results, considering a significance level of $0.05$.\\
To ensure the correctness of results, the analysis use also the \emph{Vargha and Delaney} statistic.
This test emphasize the magnitude of the differences, given by the U test, between the two algorithms.\\
The evaluation is also executed limiting the number of methods callable for each unit test, \emph{@Test}, but letting the test suite to grow in number of unit test infinitely.
The execution time is also restricted, so the program has not to run endlessly and eventually halt when it is reached the total coverage.\\
Respectively, the maximum number of method is \emph{40} while the time is \emph{60000 ms}, and given as input with the properties file~\ref{sec:prop}.\\
The next sections show the results for each application under analysis.
\section{Dimeshift}
\begin{table}[H]
	\centering
	\caption{Coverage Analysis}
	\begin{adjustbox}{max width=\textwidth}
		\begin{tabular}{|c|c|c|c|c|c|}
		\hline
			& \multicolumn{2}{c|}{Statistic} & \multicolumn{2}{c|}{Wilcoxon rank sum test} & Vargha and Delaney    \\
		\hline
			Application & Mean   & Standard Deviation   & W                               & p-value                             & A            			         \\
		\hline
			WRGen       & 35     & 6.236096             & \multirow{2}{*}{82}             & \multirow{2}{*}{0.01684}            & \multirow{2}{*}{0.82 (large)}  \\
		\cline{1-3}
			Evosuite    & 28.9   & 3.813718             &                                 &                                     &            			   		 \\
		\hline
		\end{tabular}
	\end{adjustbox}
\end{table}

\begin{table}[H]
	\centering
	\caption{Test Analysis}
	\begin{adjustbox}{max width=\textwidth}
		\begin{tabular}{|c|c|c|c|c|c|}
			\hline
			& \multicolumn{2}{c|}{Statistic} & \multicolumn{2}{c|}{Wilcoxon rank sum test} & Vargha and Delaney    \\
			\hline
			Application & Mean   & Standard Deviation   & W                               & p-value                             & A            			         \\
			\hline
			WRGen       & 24     & 1.414214             & \multirow{2}{*}{79.5}             & \multirow{2}{*}{0.02631}            & \multirow{2}{*}{0.795 (large)}  \\
			\cline{1-3}
			Evosuite    & 21.8   & 2.347576             &                                 &                                     &            			   		 \\
			\hline
		\end{tabular}
	\end{adjustbox}
\end{table}
\begin{minipage}{0.5\textwidth}
	\centering
	\emph{Coverage Analysis}\\
	\begin{adjustbox}{totalheight=125px}
		\begin{tikzpicture}
		\begin{axis}[
			boxplot/draw direction=y,
			x axis line style={opacity=0},
			axis x line*=bottom,
			axis y line=left,
			enlarge y limits,
			ymajorgrids,
			xtick={1,2},
			xticklabels={WRGen, Evosuite},
		]
		\addplot+ [
			boxplot prepared={
				lower whisker=22, lower quartile=31.25,
				median=35.50,
				upper quartile=39.75, upper whisker=42,
			},
		] coordinates {}
		node at
			(boxplot whisker cs:\boxplotvalue{lower whisker},1.1)
			{\pgfmathprintnumber{\boxplotvalue{lower whisker}}}
		node at
			(boxplot box cs: \boxplotvalue{median},1.15)
			{\pgfmathprintnumber{\boxplotvalue{median}}}
		node at
			(boxplot whisker cs:\boxplotvalue{upper whisker},1.1)
			{\pgfmathprintnumber{\boxplotvalue{upper whisker}}};
		\addplot+ [
			boxplot prepared={
				lower whisker=24, lower quartile=26.5,
				median=28,
				upper quartile=29, upper whisker=36,
			},
		] coordinates {}
		node at
			(boxplot whisker cs:\boxplotvalue{lower whisker},1.1)
			{\pgfmathprintnumber{\boxplotvalue{lower whisker}}}
		node at
			(boxplot box cs: \boxplotvalue{median},1.1)
			{\pgfmathprintnumber{\boxplotvalue{median}}}
		node at
			(boxplot whisker cs:\boxplotvalue{upper whisker},1.1)
			{\pgfmathprintnumber{\boxplotvalue{upper whisker}}};
		\end{axis}
		\end{tikzpicture}
	\end{adjustbox}
\end{minipage}
\begin{minipage}{0.5\textwidth}
	\centering
	\emph{Test Analysis}\\
	\begin{adjustbox}{totalheight=125px}
		\begin{tikzpicture}
		\begin{axis}[
		boxplot/draw direction=y,
		x axis line style={opacity=0},
		axis x line*=bottom,
		axis y line=left,
		enlarge y limits,
		enlarge x limits=0.2,
		ymajorgrids,
		xtick={1,2},
		xticklabels={WRGen, Evosuite},
		]
		\addplot+ [
		boxplot prepared={
			lower whisker=21, lower quartile=23.25,
			median=24,
			upper quartile=25, upper whisker=26,
		},
		] coordinates {}
		node at
		(boxplot whisker cs:\boxplotvalue{lower whisker},1.1)
		{\pgfmathprintnumber{\boxplotvalue{lower whisker}}}
		node at
		(boxplot box cs: \boxplotvalue{median},1.15)
		{\pgfmathprintnumber{\boxplotvalue{median}}}
		node at
		(boxplot whisker cs:\boxplotvalue{upper whisker},1.1)
		{\pgfmathprintnumber{\boxplotvalue{upper whisker}}};
		\addplot+ [
		boxplot prepared={
			lower whisker=18, lower quartile=21,
			median=21.5,
			upper quartile=23, upper whisker=26,
		},
		] coordinates {}
		node at
		(boxplot whisker cs:\boxplotvalue{lower whisker},1.1)
		{\pgfmathprintnumber{\boxplotvalue{lower whisker}}}
		node at
		(boxplot box cs: \boxplotvalue{median},1.2)
		{\pgfmathprintnumber{\boxplotvalue{median}}}
		node at
		(boxplot whisker cs:\boxplotvalue{upper whisker},1.1)
		{\pgfmathprintnumber{\boxplotvalue{upper whisker}}};
		\end{axis}
		\end{tikzpicture}

	\end{adjustbox}
\end{minipage}
\begin{minipage}{0.48\textwidth}
	\centering
	\caption{Pagekit Analysis}	
	\label{fig:pagekit}
	\begin{minipage}{0.48\textwidth}
		\caption*{Coverage}
		\begin{tikzpicture}[scale=0.43]
			\begin{axis}[
				boxplot/draw direction=y,
				x axis line style={opacity=0},
				axis x line*=bottom,
				axis y line=left,
				enlarge y limits,
				ymajorgrids,
				xtick={1,2},
				xticklabels={WRGen, Evosuite},
			]
			\addplot+ [
				boxplot prepared={
					lower whisker=16, lower quartile=17.25,
					median=19.50,
					upper quartile=21.75, upper whisker=24,
				},
			] coordinates {}
			node at
				(boxplot whisker cs:\boxplotvalue{lower whisker},1.1)
				{\pgfmathprintnumber{\boxplotvalue{lower whisker}}}
			node at
				(boxplot box cs: \boxplotvalue{median},1.15)
				{\pgfmathprintnumber{\boxplotvalue{median}}}
			node at
				(boxplot whisker cs:\boxplotvalue{upper whisker},1.1)
				{\pgfmathprintnumber{\boxplotvalue{upper whisker}}};
			\addplot+ [
				boxplot prepared={
					lower whisker=11, lower quartile=12.25,
					median=14,
					upper quartile=15.5, upper whisker=19,
				},
			] coordinates {}
			node at
				(boxplot whisker cs:\boxplotvalue{lower whisker},1.1)
				{\pgfmathprintnumber{\boxplotvalue{lower whisker}}}
			node at
				(boxplot box cs: \boxplotvalue{median},1.1)
				{\pgfmathprintnumber{\boxplotvalue{median}}}
			node at
				(boxplot whisker cs:\boxplotvalue{upper whisker},1.1)
				{\pgfmathprintnumber{\boxplotvalue{upper whisker}}};
			\end{axis}
		\end{tikzpicture}
	\end{minipage}
	\begin{minipage}{0.48\textwidth}
	\caption*{\# Tests}
		\begin{tikzpicture}[scale=0.43]
			\begin{axis}[
			boxplot/draw direction=y,
			x axis line style={opacity=0},
			axis x line*=bottom,
			axis y line=left,
			enlarge y limits,
			ymajorgrids,
			xtick={1,2},
			xticklabels={WRGen, Evosuite},
			]
			\addplot+ [
			boxplot prepared={
				lower whisker=3, lower quartile=3.25,
				median=4,
				upper quartile=4.75, upper whisker=5,
			},
			] coordinates {}
			node at
			(boxplot whisker cs:\boxplotvalue{lower whisker},1.1)
			{\pgfmathprintnumber{\boxplotvalue{lower whisker}}}
			node at
			(boxplot box cs: \boxplotvalue{median},1.15)
			{\pgfmathprintnumber{\boxplotvalue{median}}}
			node at
			(boxplot whisker cs:\boxplotvalue{upper whisker},1.1)
			{\pgfmathprintnumber{\boxplotvalue{upper whisker}}};
			\addplot+ [
			boxplot prepared={
				lower whisker=4, lower quartile=5,
				median=5,
				upper quartile=5.75, upper whisker=7,
			},
			] coordinates {}
			node at
			(boxplot whisker cs:\boxplotvalue{lower whisker},1.1)
			{\pgfmathprintnumber{\boxplotvalue{lower whisker}}}
			node at
			(boxplot box cs: \boxplotvalue{median},1.1)
			{\pgfmathprintnumber{\boxplotvalue{median}}}
			node at
			(boxplot whisker cs:\boxplotvalue{upper whisker},1.1)
			{\pgfmathprintnumber{\boxplotvalue{upper whisker}}};
			\end{axis}
		\end{tikzpicture}
	\end{minipage}
\end{minipage}
\begin{minipage}{0.48\textwidth}
	\centering
	\caption{Phoenix Analysis}
	\label{fig:phoenix}
	\begin{minipage}{0.48\textwidth}
		\caption*{Coverage}
		\begin{tikzpicture}[scale=0.43]
			\begin{axis}[
				boxplot/draw direction=y,
				x axis line style={opacity=0},
				axis x line*=bottom,
				axis y line=left,
				enlarge y limits,
				ymajorgrids,
				xtick={1,2},
				xticklabels={WRGen, Evosuite},
			]
			\addplot+ [
				boxplot prepared={
					lower whisker=39, lower quartile=58,
					median=59.50,
					upper quartile=65.25, upper whisker=68,
				},
			] coordinates {}
			node at
				(boxplot whisker cs:\boxplotvalue{lower whisker},1.1)
				{\pgfmathprintnumber{\boxplotvalue{lower whisker}}}
			node at
				(boxplot box cs: \boxplotvalue{median},1.15)
				{\pgfmathprintnumber{\boxplotvalue{median}}}
			node at
				(boxplot whisker cs:\boxplotvalue{upper whisker},1.1)
				{\pgfmathprintnumber{\boxplotvalue{upper whisker}}};
			\addplot+ [
				boxplot prepared={
					lower whisker=32, lower quartile=43.25,
					median=50,
					upper quartile=52.25, upper whisker=58,
				},
			] coordinates {}
			node at
				(boxplot whisker cs:\boxplotvalue{lower whisker},1.1)
				{\pgfmathprintnumber{\boxplotvalue{lower whisker}}}
			node at
				(boxplot box cs: \boxplotvalue{median},1.1)
				{\pgfmathprintnumber{\boxplotvalue{median}}}
			node at
				(boxplot whisker cs:\boxplotvalue{upper whisker},1.1)
				{\pgfmathprintnumber{\boxplotvalue{upper whisker}}};
			\end{axis}
		\end{tikzpicture}
	\end{minipage}
	\begin{minipage}{0.48\textwidth}
	\caption*{\# Tests}
		\begin{tikzpicture}[scale=0.43]
			\begin{axis}[
			boxplot/draw direction=y,
			x axis line style={opacity=0},
			axis x line*=bottom,
			axis y line=left,
			enlarge y limits,
			ymajorgrids,
			xtick={1,2},
			xticklabels={WRGen, Evosuite},
			]
			\addplot+ [
			boxplot prepared={
				lower whisker=7, lower quartile=9,
				median=9,
				upper quartile=9.75, upper whisker=11,
			},
			] coordinates {}
			node at
			(boxplot whisker cs:\boxplotvalue{lower whisker},1.1)
			{\pgfmathprintnumber{\boxplotvalue{lower whisker}}}
			node at
			(boxplot box cs: \boxplotvalue{median},1.15)
			{\pgfmathprintnumber{\boxplotvalue{median}}}
			node at
			(boxplot whisker cs:\boxplotvalue{upper whisker},1.1)
			{\pgfmathprintnumber{\boxplotvalue{upper whisker}}};
			\addplot+ [
			boxplot prepared={
				lower whisker=7, lower quartile=10.25,
				median=11,
				upper quartile=11, upper whisker=12,
			},
			] coordinates {}
			node at
			(boxplot whisker cs:\boxplotvalue{lower whisker},1.1)
			{\pgfmathprintnumber{\boxplotvalue{lower whisker}}}
			node at
			(boxplot box cs: \boxplotvalue{median},1.1)
			{\pgfmathprintnumber{\boxplotvalue{median}}}
			node at
			(boxplot whisker cs:\boxplotvalue{upper whisker},1.1)
			{\pgfmathprintnumber{\boxplotvalue{upper whisker}}};
			\end{axis}
		\end{tikzpicture}
	\end{minipage}
\end{minipage}
\begin{minipage}{0.48\textwidth}
	\centering
	\caption{Retroboard Analysis}
	\label{fig:retroboard}
	\begin{minipage}{0.48\textwidth}
		\caption*{Coverage}
		\begin{tikzpicture}[scale=0.43]
			\begin{axis}[
				boxplot/draw direction=y,
				x axis line style={opacity=0},
				axis x line*=bottom,
				axis y line=left,
				enlarge y limits,
				enlarge x limits=0.2,
				ymajorgrids,
				xtick={1,2},
				xticklabels={WRGen, Evosuite},
			]
			\addplot+ [
				boxplot prepared={
					lower whisker=62, lower quartile=66,
					median=70.5,
					upper quartile=72, upper whisker=76,
				},
			] coordinates {}
			node at
				(boxplot whisker cs:\boxplotvalue{lower whisker},1.2)
				{\pgfmathprintnumber{\boxplotvalue{lower whisker}}}
			node at
				(boxplot box cs: \boxplotvalue{median},1.25)
				{\pgfmathprintnumber{\boxplotvalue{median}}}
			node at
				(boxplot whisker cs:\boxplotvalue{upper whisker},1.2)
				{\pgfmathprintnumber{\boxplotvalue{upper whisker}}};
			\addplot+ [
				boxplot prepared={
					lower whisker=52, lower quartile=52,
					median=53.5,
					upper quartile=55, upper whisker=59,
				},
			] coordinates {}
			node at
				(boxplot whisker cs:\boxplotvalue{lower whisker},1.35)
				{\pgfmathprintnumber{\boxplotvalue{lower whisker}}}
			node at
				(boxplot box cs: \boxplotvalue{median},1.25)
				{\pgfmathprintnumber{\boxplotvalue{median}}}
			node at
				(boxplot whisker cs:\boxplotvalue{upper whisker},1.2)
				{\pgfmathprintnumber{\boxplotvalue{upper whisker}}};
			\end{axis}
		\end{tikzpicture}
	\end{minipage}
	\begin{minipage}{0.48\textwidth}
	\caption*{\# Tests}
		\begin{tikzpicture}[scale=0.43]
			\begin{axis}[
			boxplot/draw direction=y,
			x axis line style={opacity=0},
			axis x line*=bottom,
			axis y line=left,
			enlarge y limits,
			enlarge x limits=0.2,
			ymajorgrids,
			xtick={1,2},
			xticklabels={WRGen, Evosuite},
			]
			\addplot+ [
			boxplot prepared={
				lower whisker=22, lower quartile=25,
				median=26,
				upper quartile=26.75, upper whisker=30,
			},
			] coordinates {}
			node at
			(boxplot whisker cs:\boxplotvalue{lower whisker},1.1)
			{\pgfmathprintnumber{\boxplotvalue{lower whisker}}}
			node at
			(boxplot box cs: \boxplotvalue{median},1.15)
			{\pgfmathprintnumber{\boxplotvalue{median}}}
			node at
			(boxplot whisker cs:\boxplotvalue{upper whisker},1.1)
			{\pgfmathprintnumber{\boxplotvalue{upper whisker}}};
			\addplot+ [
			boxplot prepared={
				lower whisker=22, lower quartile=25.25,
				median=26.5,
				upper quartile=27, upper whisker=30	,
			},
			] coordinates {}
			node at
			(boxplot whisker cs:\boxplotvalue{lower whisker},1.2)
			{\pgfmathprintnumber{\boxplotvalue{lower whisker}}}
			node at
			(boxplot box cs: \boxplotvalue{median},1.25)
			{\pgfmathprintnumber{\boxplotvalue{median}}}
			node at
			(boxplot whisker cs:\boxplotvalue{upper whisker},1.1)
			{\pgfmathprintnumber{\boxplotvalue{upper whisker}}};
			\end{axis}
		\end{tikzpicture}
	\end{minipage}
\end{minipage}
\section{Splittypie}
\begin{table}[H]
	\centering
	\caption{Coverage Analysis}
	\begin{adjustbox}{max width=\textwidth}
		\begin{tabular}{|c|c|c|c|c|c|}
		\hline
			& \multicolumn{2}{c|}{Statistic} & \multicolumn{2}{c|}{Wilcoxon rank sum test} & Vargha and Delaney    \\
		\hline
			Application & Mean   & Standard Deviation   & W                               & p-value                             & A            			         \\
		\hline
			WRGen       & 40.4   & 4.526465             & \multirow{2}{*}{96}             & \multirow{2}{*}{0.0005501}          & \multirow{2}{*}{0.96 (large)}  \\
		\cline{1-3}
			Evosuite    & 26.2   & 7.568942             &                                 &                                     &            			   		 \\
		\hline
		\end{tabular}
	\end{adjustbox}
\end{table}

\begin{table}[H]
	\centering
	\caption{Test Analysis}
	\begin{adjustbox}{max width=\textwidth}
		\begin{tabular}{|c|c|c|c|c|c|}
			\hline
			& \multicolumn{2}{c|}{Statistic} & \multicolumn{2}{c|}{Wilcoxon rank sum test} & Vargha and Delaney    \\
			\hline
			Application & Mean   & Standard Deviation   & W                               & p-value                             & A            			         \\
			\hline
			WRGen       & 17.2   & 3.224903             & \multirow{2}{*}{6.5}            & \multirow{2}{*}{0.0008922}          & \multirow{2}{*}{0.065 (large)}  \\
			\cline{1-3}
			Evosuite    & 21.9   & 2.078995             &                                 &                                     &            			   		 \\
			\hline
		\end{tabular}
	\end{adjustbox}
\end{table}
\begin{minipage}{0.5\textwidth}
	\centering
	\emph{Coverage Analysis}\\
	\begin{adjustbox}{totalheight=125px}
		\begin{tikzpicture}
		\begin{axis}[
			boxplot/draw direction=y,
			x axis line style={opacity=0},
			axis x line*=bottom,
			axis y line=left,
			enlarge y limits,
			ymajorgrids,
			xtick={1,2},
			xticklabels={WRGen, Evosuite},
		]
		\addplot+ [
			boxplot prepared={
				lower whisker=34, lower quartile=36.75,
				median=40,
				upper quartile=44, upper whisker=48,
			},
		] coordinates {}
		node at
			(boxplot whisker cs:\boxplotvalue{lower whisker},1.1)
			{\pgfmathprintnumber{\boxplotvalue{lower whisker}}}
		node at
			(boxplot box cs: \boxplotvalue{median},1.15)
			{\pgfmathprintnumber{\boxplotvalue{median}}}
		node at
			(boxplot whisker cs:\boxplotvalue{upper whisker},1.1)
			{\pgfmathprintnumber{\boxplotvalue{upper whisker}}};
		\addplot+ [
			boxplot prepared={
				lower whisker=16, lower quartile=19.25,
				median=26,
				upper quartile=32, upper whisker=39,
			},
		] coordinates {}
		node at
			(boxplot whisker cs:\boxplotvalue{lower whisker},1.1)
			{\pgfmathprintnumber{\boxplotvalue{lower whisker}}}
		node at
			(boxplot box cs: \boxplotvalue{median},1.1)
			{\pgfmathprintnumber{\boxplotvalue{median}}}
		node at
			(boxplot whisker cs:\boxplotvalue{upper whisker},1.1)
			{\pgfmathprintnumber{\boxplotvalue{upper whisker}}};
		\end{axis}
		\end{tikzpicture}
	\end{adjustbox}
\end{minipage}
\begin{minipage}{0.5\textwidth}
	\centering
	\emph{Test Analysis}\\
	\begin{adjustbox}{totalheight=125px}
		\begin{tikzpicture}
		\begin{axis}[
		boxplot/draw direction=y,
		x axis line style={opacity=0},
		axis x line*=bottom,
		axis y line=left,
		enlarge y limits,
		ymajorgrids,
		xtick={1,2},
		xticklabels={WRGen, Evosuite},
		]
		\addplot+ [
		boxplot prepared={
			lower whisker=12, lower quartile=14.25,
			median=19,
			upper quartile=19, upper whisker=21,
		},
		] coordinates {}
		node at
		(boxplot whisker cs:\boxplotvalue{lower whisker},1.1)
		{\pgfmathprintnumber{\boxplotvalue{lower whisker}}}
		node at
		(boxplot box cs: \boxplotvalue{median},1.15)
		{\pgfmathprintnumber{\boxplotvalue{median}}}
		node at
		(boxplot whisker cs:\boxplotvalue{upper whisker},1.1)
		{\pgfmathprintnumber{\boxplotvalue{upper whisker}}};
		\addplot+ [
		boxplot prepared={
			lower whisker=19, lower quartile=21,
			median=21,
			upper quartile=23.5, upper whisker=25,
		},
		] coordinates {}
		node at
		(boxplot whisker cs:\boxplotvalue{lower whisker},1.1)
		{\pgfmathprintnumber{\boxplotvalue{lower whisker}}}
		node at
		(boxplot box cs: \boxplotvalue{median},1.1)
		{\pgfmathprintnumber{\boxplotvalue{median}}}
		node at
		(boxplot whisker cs:\boxplotvalue{upper whisker},1.1)
		{\pgfmathprintnumber{\boxplotvalue{upper whisker}}};
		\end{axis}
		\end{tikzpicture}

	\end{adjustbox}
\end{minipage}
\newpage