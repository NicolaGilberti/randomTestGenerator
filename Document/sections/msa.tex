\section{Master-slave architecture }\label{sec:MSArch}
The program performs different steps to achieve the final result and each step is performed by a different \textit{worker}. Specifically, the program uses two distinct threads. This architecture, that other test case generators (as Evosuite~\cite{evosuite}) implement, can be used to balance the work-load on different processors to increase the efficiency.
 
%Although EvoSuite is a better choice because it perform more different operations, some of them are not required for the scope of the project.Thus, it is demanded a resource affordable project that perform only the required operations, avoiding not necessary tasks.

The algorithm use threads for specific purposes, defining distinct operations and scopes for each of them. Firstly the main thread start, managing all the imports from resources (see \autoref{sec:prop}) to prepare the environment for the other thread. In fact, the main thread updates dynamically the Java \textit{classpath} in order to recover all the classes needed for the runtime execution.

The second thread (slave) is activated by the main thread (master) once the environment is set up. The second thread generates and executes test cases and sends messages to the main thread with the results of its computations.

The second thread instruments the input class, a Java class containing Page Object methods for a given web application, and generates tests as sequences of those methods (see \autoref{sec:tg}). Those tests are sent back to the primary thread through a \textit{LinkedBlockingQueue}~\cite{linkedBlockingQueue}, that checks their validity and creates a well-formed test suite. The choice of a LinkedBlockingQueue is a default option to reduce the impact on the memory load during inactivity and to avoid errors due to concurrency at run-time. \hl{Even if there can't be simultaneous access from both threads, because the second one generates the message and it restarts creating a new one, the implementation follows the best-practices hints for correct data exchange between threads}.