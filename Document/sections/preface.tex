\chapter*{Abstract\markboth{Abstract}{Abstract}}\label{ch:abstract}
\addcontentsline{toc}{chapter}{Abstract}
%With the high number of software that nowadays is in commerce, testing has an important role for companies.
%In fact, testing is necessary to evaluate the quality of the software.\\
%However, thanks to the new technologies, other problems can be \emph{translated} into code and tested accordingly.\\
%Specifically, Web-pages testing can be done representing all the executable path through a graph.
%Graph where each node is a state of the application and the transitions are the links, buttons, text-boxes, operations that lead the service from a state to another.\\
%Although now the problem can be solved with known technologies, web-pages can be complicated so a hands-on approach for the test generation is expensive.\\
%Hence, it is required an automatic way to generate tests.\\
%This program is able to solve the problem creating a testsuite for the input application following the random testing approach.\\
%The goal is performed operating on the tested application.
%Firstly an instrumentation phase to insert extra code for the analysis.
%Then, the creation of test-cases for the line-coverage\textbackslash branch-coverage analysis is performed.\\
%The program is based on Java and the instrumentation is for Java classes.\\


Nowadays, software has a big impact in all aspects of our society. More and more companies develop software and testing plays a crucial role for ensuring its quality.

%Thanks to new technologies, which are continuously facing those problems, other scenarios can be explored and tested accordingly using the code (to be modified). More specifically, the testing phase on web pages can be carried out making use of a graph, where each depicted node represents a state of the application, transitions correspond to links, buttons, text boxes or every kind of operation which can lead to a movement from an application state to another.
%
%Despite the solution to the illustrated situation can be exploited with other SoA processes, web pages can be extremely demanding in terms of test generation.

Web applications present new testing challenges with respect to traditional software testing, due to their dynamic and heterogeneous nature. In this document, a tool (Web Random Generator) able to create a test suite for a given application following a random testing approach is presented. WRGen first instruments Page Objects developed for the given application to measure coverage. Then it generates random test cases as sequence of Page Object methods.

Finally, a comparison with a state-of-the-art random generator tool (Evosuite) on 5 web applications is carried out. In terms of coverage, WRGen outperforms Evosuite random generator over all subjects.

\vfill\noindent
\hfill
\begin{flushright}
\begin{minipage}[b]{0.45\textwidth}
 \centering
  Nicola Gilberti\\
 {\footnotesize <nicola.gilberti \textit{at} studenti.unitn.it>}
\end{minipage}
\end{flushright}
\newpage