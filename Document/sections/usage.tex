\chapter{Usage}\label{ch:usage}

This chapter explains all the settings that are necessary for the project to run. More specifically, it shows the \textit{Maven} \textit{POM} file and the properties file needed to set up the environment correctly. The project on \href{\projRoot}{Github} also contains a bash script with an example of how the program can be run.

\section{Project Object Model}\label{sec:pom}

The POM contains information about the project and various configuration details used by Maven to build the project. To compile classes, a specific \emph{JDK} library has to be imported as a dependency inside the project (\hl{If you mean the classes of the Java project where is located the class under test, they should be already compiled}).

Because it depends on \emph{JDK} and so on the machine architecture, a \emph{<profile>} tag is necessary to identify all the possible options, to maintain the code machine independent.
\begin{lstlisting}[language=XML,caption={Windows default profile for jdk lib},label={lst:winToolsJar}]% Start your code-block

<profile>
	<id>windows_profile</id>
	<activation>
		<os>
			<family>Windows</family>
		</os>
	</activation>
	<properties>
		<toolsjar>${java.home}\..\lib\tools.jar</toolsjar>
	</properties>
</profile>
\end{lstlisting}

There are specific paths for each OS type, and Maven accepts only specific \href{http://maven.apache.org/enforcer/enforcer-rules/requireOS.html}{\textit{OS families}}.

(\hl{Overall, it is not clear why you need the tools.jar library}).
The tag \textit{<toolsjar>} is an ad-hoc tag created to define a new property, in this case the path to the tools.jar library. In \autoref{lst:instToolsJar}, there are a few elements to consider. First, the \textit{phase} in which the installation of this library has to be performed that is, in this case, the \textit{install} phase. Second, the \textit{configuration} of the library (\hl{which one tools.jar?}) is based on the one (\hl{which library, JDK?}) installed on your machine, but in this case the \emph{JDK} library has no different option if the OS changes (\hl{so why is it important this configuration tag?}). \hl{The version is the value that the runner of the program \emph{MUST} verify}.

\begin{lstlisting}[language=XML,caption={How to 'mvn install' the tools.jar},label={lst:instToolsJar}]% Start your code-block

<plugin>
	<groupId>org.apache.maven.plugins</groupId>
	<artifactId>maven-install-plugin</artifactId>
	<version>2.5.2</version>
	<executions>
		<execution>
			<id>install-external</id>
			<phase>install</phase>
			<configuration>
				<file>${toolsjar}</file>
				<repositoryLayout>default</repositoryLayout>
				<groupId>com.sun</groupId>
				<artifactId>tools</artifactId>
				<version>1.8.0</version>
				<packaging>jar</packaging>
				<generatePom>true</generatePom>
			</configuration>
			<goals>
				<goal>install-file</goal>
			</goals>
		</execution>
	</executions>
</plugin>
\end{lstlisting}

\section{Properties}\label{sec:prop}
A .properties is a file extension for files used in Java programs to specify the configurable parameters of a Java application. WRGen requires information about the specific class under test that has to analyze. More specifically, a list of all the properties that are configurable in WRGen is shown below:

\begin{itemize}
	\item \textbf{FileName}: the \textit{complete} name (package names included) of the class under test, is required.	E.g.: FileName = com.main.ClassUnderTest;
	\item \textbf{ProjectName}: the name of the java project that contains the class under test is required;
	\item \textbf{PathToProjectDir}: the absolute path of the java project that contains the class under test;
	\item \textbf{LineToCover}: colon separated list of numbers representing the lines to cover in the class under test. E.g.:LineToCover = 33:47:60:72:85:99:115:132:144:157 \hl{in the example Listing 2.10 you talk about branches, here about lines. Change the example above to be consistent.}
	\item \textbf{Separator}: in the LineToCover property the default separator is the colon, but a different one can be used which can be specified in this property;
	\item \textbf{ExecutionTestTimer}: timeout in milliseconds for the test generation phase of WRGen;
	\item \textbf{MaxNumberOfMethodXTest}: each test case has a random length. In this property the user specifies the maximum length of a test case;
	\item \textbf{GraphName}: name of the web application graph ;
	\item \textbf{GraphDirPath}: absolute path of the dot file containing the web application graph;
	\item \textbf{StartNodeGraph}: name of the node in the web application graph modelling the home page of the web application under test;
	\item \textbf{RequiredPath}: The program use reflection to execute the external project written in the first item, but it is not able to know all the dependencies it need. (\hl{is the classpath of the Java project containing the class under test?})
\end{itemize}