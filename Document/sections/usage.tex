\chapter{Usage}\label{ch:usage}
This chapter explain all the features that are necessary for the project to run.\\
Specifically it will show some \emph{not} well-known tags in the POM, and the properties file necessary for the Environment.\\
The project on \href{\projRoot}{Github} contains also a .sh script with an example of how the program can be runned.
\section{Project Object Model}\label{sec:pom}
The POM contains information about the project and various configuration detail used by Maven to build the project.\\
To compile classes at runtime, a specific \emph{JDK} library has to be imported as a dependency inside the project.\\
Because it depends on \emph{JDK} and so on the machine architecture, a \emph{<profile>} tag is necessary to identify all the possible options, to maintain the code independent from the machine.
\begin{lstlisting}[language=XML,caption={Windows default profile for jdk lib},label={lst:winToolsJar}]% Start your code-block

<profile>
	<id>windows_profile</id>
	<activation>
		<os>
			<family>Windows</family>
		</os>
	</activation>
	<properties>
		<toolsjar>${java.home}\..\lib\tools.jar</toolsjar>
	</properties>
</profile>
\end{lstlisting}
There are specific path for each OS type, and Maven accept only specific \href{http://maven.apache.org/enforcer/enforcer-rules/requireOS.html}{\emph{OS family}}.\\
\emph{<toolsjar>} is an ad-hoc tag created to define a new property, in this case the path to the tools.jar lib.\\
This variable will be used during a plugin execution.\\
In the next listing~\ref{lst:instToolsJar}, there are few element to consider.
The phase in which this operation work is decidable by the programmer, in this case the \emph{install} phase is selected.\\
The configuration of the library is based on the one installed on your machine, but in this case the \emph{JDK} library has no different option if the OS change.\\
The version is the value that the runner of the program \emph{MUST} verify.
\begin{lstlisting}[language=XML,caption={How to 'mvn install' the tools.jar},label={lst:instToolsJar}]% Start your code-block

<plugin>
	<groupId>org.apache.maven.plugins</groupId>
	<artifactId>maven-install-plugin</artifactId>
	<version>2.5.2</version>
	<executions>
		<execution>
			<id>install-external</id>
			<phase>install</phase>
			<configuration>
				<file>${toolsjar}</file>
				<repositoryLayout>default</repositoryLayout>
				<groupId>com.sun</groupId>
				<artifactId>tools</artifactId>
				<version>1.8.0</version>
				<packaging>jar</packaging>
				<generatePom>true</generatePom>
			</configuration>
			<goals>
				<goal>install-file</goal>
			</goals>
		</execution>
	</executions>
</plugin>
\end{lstlisting}
\section{Properties}\label{sec:prop}
A .properties is a file extension for files mainly used in Java related technologies to store the configurable parameters of an application.\\
This project require information about the project that has to analyse and about the specific class to instrument.
\begin{itemize}
	\item \textbf{FileName}:\\
	It is required the name of the Class to test, with the complete package route.\\
	E.g.:FileName = com.main.ClassUnderTest
	\item \textbf{ProjectName}:\\
	It is required the name of the Project that contains the class.
	\item \textbf{PathToProjectDir}:\\
	The program has to know where the project you are referring to is placed.
	\item \textbf{LineToCover}:\\
	The line-coverage has to know which line to instrument.\\
	E.g.:LineToCover = 33:47:60:72:85:99:115:132:144:157
	\item \textbf{Separator}:\\
	Lines in the previous tag are a list of line numbers divided by a separator.\\
	The default accepted is ':' but it can be used a different one defining it here.
	\item \textbf{ExecutionTestTimer}:\\
	This is the time the algorithm has to find tests.
	It is in millisecond, so 60000 is equal to 1 minute.
	\item \textbf{MaxNumberOfMethodXTest}:\\
	Each test can contain a random number of method calls. This attribute is an upper bound to it.
	\item \textbf{GraphName}:\\
	As seen in~\ref{sec:tg}, this project work generating a sequence of method following a graph.
	\item \textbf{GraphDirPath}:\\
	The program has to know where the graph you are referring to is placed.
	\item \textbf{StartNodeGraph}:\\
	It is the starting point of the graph path analysis.
	\item \textbf{RequiredPath}:\\
	The program use reflection to execute the external project written in the first item, but it is not able to know all the dependencies it need.
\end{itemize}